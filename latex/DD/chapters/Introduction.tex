\section{Purpose}
In this document we aim to provide a description for the architecture and design of MyTaxiService.
This document is targeted towards the future developers of the system.

\section{Scope}
The application will be developed using a client-server paradigm.
The server-side application must recognize an user (either an unregistered guest, a passenger, a taxi driver or an administrator), 
and accordingly signal the client the available actions. \\
The server-side application must manage the city-wide taxi deployment, by the means explained in the RASD 
document\footnote{see reference documents}. \\ The client-side application must show a UI to which the users can interact. \\
The application must implement a report system, in order to incentive the good behavior of the users involved.

\section{Definition, Acronyms, Abbreviation}
Additional definitions can be retrieved from the corresponding ``Definition, Acronyms, Abbreviation'' section in the RASD document.
\begin{itemize}
 \item RASD: Requirements Analysis and System Design
 \item API: Application Programming Interface
 \item UI: User Interface
 \item MVC: Model View Controller
 \item Client-side: client application, front end
 \item Server-side: server application, back end
\end{itemize}

\section{Reference Documents}
In order to write this, we referred to the following documents:
\begin{itemize}
 \item [1]: our RASD document
 \item [2]: the design document template received from the professor
\end{itemize}

\section{Document Structure}
\begin{itemize} 
 \item[Architectural design]: here we describe every components of our system, and their functions.
 \item[Algorithm design]: here we define which algorithms to outsource, which ones to develop ad-hoc, 
 and we give a brief overview of the latter ones.
 \item[User interface design]: here we write some guidlines in order to design a better user interface.
 \item[Requirements traceability]:here we describe how our components interact, in order to meet the equirements.
\end{itemize}
