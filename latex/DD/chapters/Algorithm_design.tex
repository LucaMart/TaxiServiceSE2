\section{Outsourced algorithms}
% Punti principali:
% non reinventare la ruota
% * usiamo servizi esistenti per il mapping / pathfinding
% --> "vedi Component interfaces"
Routing and path-finding are critical components of our infrastructure; therefore, we have determined
that the best approach would be to resort to domain-specific, well tested third party solutions.
The developer should implement an adapter to the mapping service, since some experimentation will
be required in order to find the solution which best suits our specific needs. For
the same reason, we do not push the choice of a specific routing API; however, some criteria
have to be satisfied:
\begin{itemize}
 \item The API must answer our request withing a reasonable time (doesn't violate the timing constraints specified in RASD)
 \item The API must take advantage of updated traffic information
\end{itemize}
Some obvious API choices would be Google's and MapQuest's


\section{In-house developed algorithms}

\subsection{Shared ride}
The following pseudocode describes how share taxi request should be handled
\begin{algorithmic}
 \Function {sharedRequest} {passenger}
  \State $ shared \gets findSharedRideAvailable(fromZone, toZone, timeout) $
  \If {exists shared}
    \State $ shared.addReservation() $
  \Else
    \State $ taxi \gets getAvailableTaxi(fromZone) $
    \If {exists $ taxi $}
      \State create new shared ride
    \Else
      \State error message
    \EndIf
  \EndIf
 \EndFunction
\end{algorithmic}
\vspace{5mm}

The following function (dependency of \textit{sharedRequest}), describes how the system should search for an
available shared ride, compatible with the constraints provided via arguments
\begin{algorithmic}
 \Function {findSharedRideAvailable} {fromZone, toZone, timeout}
  \ForAll{ride in Rides}
    \If {ride is shared}
      \If {$ride.fromZone == fromZone$ AND $ride.toZone == toZone$}
	\If {not ride.isFull() AND not ride.isReserved()}
	  \If {$now() - ride.allocationTime < timeout$}
	    \State \Return ride
	  \EndIf
	\EndIf
      \EndIf
    \EndIf
  \EndFor
  \State \Return empty set
 \EndFunction
\end{algorithmic}
\vspace{5mm}

The following function (dependency of \textit{sharedRequest}), describes how the system should fetch an available taxi.
Keep in mind that the process of issuing the call to a driver and managing his/her response happens inside this function
(or its sub-routines). \\
Note: a timeout for the ``accept/refuse call'' action (executed by the taxi driver) must be implemented. However, it
should be encapsulated inside taxi.sendRequest(), rather that the high level getAvailableTaxi() function.
\begin{algorithmic}
\Function {getAvailableTaxi} {fromZone}
\ForAll {queue in Queues}
  \If { queue.getZone()==fromZone AND not queue.isEmpty()}
       \State $ taxi \gets queue.getFirstTaxi() $
     \If {taxi.sendRequest() == SUCCESS}
         \State manage queue
         \Return taxi
     \Else 
         \State manage queue and retry
    \EndIf
\EndIf
\EndFor
\EndFunction


\end{algorithmic}

\subsection{Billing calculation}
The following formula describes how the system should calculate the amount of money each passenger
has to pay after a shared ride (to put it simply, how to ``split the bill'') % Biiiilll!
\begin{align*}
 B_{i} &= \text{Amount paid by the i-th passenger} \\
 D_{i} &= \text{Distance traveled by the i-th passenger} \\
 D &= \text{Total traveled distance} \\
 C &= \text{Total amount calculated by the taximeter} \\
 B_{i} &= \frac{D_{i}}{D} * C |_{\substack{\text{rounded to the nearest €0.1 }}} \\
 T &= \sum_{i=0}^{n} B_{i} = \text{Total cost, calculated as the sum of the n partial costs}
\end{align*}