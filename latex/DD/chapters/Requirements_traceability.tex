
\begin{itemize}
 \item[G1]: clicking on the button contained in the passenger home activity, a taxi request form will be shown. 
  Filling the form allows the user to make a taxi request.
 \item[G2]: as soon as a taxist decides to take care about a call, the controller asks the map service to calculate an ETA. 
 After that, the system will notify the passenger's client, that will display those information through an ad hoc activity.
 \item[G3]: when a taxi driver accepts to take care about a call, his controller will remove him from the queue.
 \item[G4]: in order to request a shared ride, a passenger must fill the form for a reservation/request, and add the shared option;
 The server will create a ride object, search for a sharedride that respect the constraints explained in the RASD document,
 and add the ride object to the shared ride; otherwise a new sharedride will be created.
 \item[G5]: when the taxi driver's client receive a call notification from the server, the application will select and display 
 and ad hoc activity, that allow the taxi driver to click on "accept" or "refuse". After that, the client will send a message to the
 server, in order to inform him about the human's decision.
 \item[G6]: the taxi driver home activity provide a button that switch the taxist status; it will requests the controller to 
 remove/ add himself to a taxi queue.
 \item[G7]: when a taxists accepts to take care about a requests ( either shared or not), the controller will notify the client about
 the ETA and the fee amount. The client will display an ad hoc activity for these informations. 
 \item[G8]: a web-based, http-based API will be released to aid future improvements
\end{itemize}

\begin{center}
 \begin{longtable}{| m{2cm} | m{6cm} | m{5cm} | }
 \hline
  Goal & Requirement & Design document sections \newline involved\\
 \hline\hline
 \endhead
   G1&	Persons can create an account& 2.3.1 Userinterface, Action, Activity, Clientnetworkinterface\newline
					2.3.2 Controller, User, Servernetworkinterface\newline
					2.3.4 Database server\\ \hline
   G1&	Persons can log into their account& 2.3.1 Userinterface, Action, Activity, Clientnetworkinterface\newline
					      2.3.2 Controller, User, Servernetworkinterface\newline
					      2.3.4 Database server\\ \hline
   G1&	Passengers can select from a menu the option of requesting a taxi as soon as possible& 2.3.1 Userinterface, Action, Activity\\ \hline
   G1&	Passengers can insert their position filling an input form and confirm it& 2.3.1 Userinterface, Action, Activity, Clientnetworkinterface\newline 
										    2.1 GPS \\ \hline
   G1&	The system will receive the request and identify the zone in which the passenger is in	& 2.3.2 Servernetworkinterface,Ride, Controller \\ \hline 
   G1&	The system will forward the request to the first taxi in the selected zone queue and wait for an answer	& 2.3.2 Controller, Ridesmanager, Servernetworkinterface\\ \hline
   G1&	If the taxist accepts, the system will remove him from the queue; otherwise it will append the taxist to the last position and
scan the list for a taxist to accept& 2.3.2 Controller, Ridesmanager, Ride\\ \hline
    G2&	As soon as a taxist accepts a request, the system invokes the support system to calculate the ETA giving the position of the 
	taxi and the position of the passenger& 2.3.2 Servernetworkinterface, Controller, Ridesmanager, Ride \newline
						2.5 Google map service\\ \hline
    G2& he system will communicate the taxi code and the ETA& 2.3.1 Userinterface, Activity, Clientnetworkinterface\newline
								2.3.2 Controller, Servernetworkinterface, Ridesmanager, Ride\\ \hline
    G3& Passengers can select from a menu the option of reserving a taxi for a chosen ride and date & 2.3.1 Userinterface, Activity, Action \\ \hline
    G3& Passengers can insert the initial and final position, time and date, their email and confirm it & 2.3.1 Userinterface, Activity, Action \newline 2.1 GPS\\ \hline
    G3&The system will receive the reservation and if it respects the 2 hour constraint it will send a confirmation & 2.3.2 Servernetworkinterface, Controller, Ridesmanager, Ride \\ \hline
    G3&Ten minutes before the ride starts, the system allocates a taxi for it.  & 2.3.2 Passenger, Taxidriver, Ridesmanager, Ride\\ \hline
    G4&The application must have a selectable option labled:"share your ride", that allows passengers to enable the shared 
	ride service. In case of non reserved ride, the application will ask passengers the amount of time they can wait for others people  & 2.3.1 Userinterface, Activity, Action \\ \hline
    G4& When the system receive a request of a shared ride, it will search for others shared ride requests starting from the same
	taxi zone, and going in the same direction & 2.3.2 Servernetworkinterface, Controller, Ridesmanager, Sharedride, Ride\newline
						      3.2.1 sharedRequest, findSharedRideAvailable\\ \hline
    G4& When a new passenger is added to a shared ride, the system will interact with the map service, in order to 
	retrieve a new route for the taxi driver, and to calculate new fees & 2.3.2 Controller, Ridesmanager, Ride, Sharedride\newline
										2.5 Google map service\newline
										3.2.2 Billing calculation\\ \hline
    G4& When the timeout of one passengers ,added to the current ride, occur, the system will procede with the allocation of the taxi & 2.3.2 Ridesmanager, Sharedride, Ride\newline
																	  3.2.1 getAvailableTaxi\\ \hline
    G4& After the taxi allocation, the passengers who requested the shared ride will receive, not only the taxi ID, but also 
	the fee they have to pay & 2.3.1 Userinterface, Activity, Clientnetworkinterface\\ \hline
    G5& The system must forward a taxi request in the following cases:
	  \begin{itemize}
	   \item [1:] A passenger has requested a ride.
	   \item [2:] A taxi reservation is sheduled to begin in 10 minutes.
	  \end{itemize} & 2.3.2 Controller, Ridesmanager, Ride, Servernetworkinterface\\ \hline
    G5& If a taxi driver refuses to take care about a call, the system will move him at the end of the queue,and forward the
	request to the next taxi driver in the queue. If a queue is empty, the system will notify the passenger that there are no taxi available& 2.3.2	Servernetworkinterface, Controller, Ridesmanager\\ \hline
    G5&If a taxe driver accepts to take care of the call, the system shall  remove him from the queue. & 2.3.2 Servernetworkinterface, Controller, Ridesmanager\\ \hline
    G6&A taxi driver logged in into the system can select the button " Ready ", then the application will notify the server that 
	the logged user is ready to accept some passengr's call. The application also send the taxi driver's position detected with a GPS& 2.3.1 Userinterface, Activity, Action, Clientnetworkinterface\newline
																	    2.3.2 Servernetworkinterface, Controller, Ridesmanager\newline
																	     2.1 GPS\\ \hline
    G6&If the application needs to retrieve data from a GPS and this isn't available, it will remind  the user to turn it on&  2.3.1 Activity, Userinterface, Action\\ \hline
    G6&When the system receive a notification , by a taxi driver, informing that he is ready to take care of some passengers, 
	it will append the user in the queue corresponding to the taxi zone that include the position retrieved by the application&  2.3.2 Servernetworkinterface, Controller,Ridesmanager\\ \hline
    G7&When a taxi driver is assigned to a shared ride, the system will send him the route he needs to follow, and the fee 
	amount every passenger have to pay& 2.3.2 Controller, Ridesmanager, Sharedride, Servernetworkinterface\\ \hline
    G7&When a driver is assigned to a non-shared ride, the system will send him the route he needs to follow, and the fee
	amount the passenger has to pay& 2.3.2 Controller, Ridesmanager, Ride, Servernetworkinterface \\ \hline
    G8& It is also neccessary to develop programmatic interfaces that allow to customize the system, adding new features& 2.5.3 In-house developed API\\ \hline
      & Passengers can access a section, in which they be able to check the ID of the taxi assigned to their ride 
	and manage ( delete or modify) an active reservation&  2.3.1 Userinterface, Action, Activity\\ \hline
      & When a passenger delete a reservation , the system will remove it from the reservation scheduler and, if a taxi 
	driver is already assigned, notify the taxist& 2.3.1 Userinterface, Activity, Clientnetworkinterface\newline
							2.3.2 Controller, Servernetworkinterface, Ridesmanager, Ride\\ \hline
      & A passenger can modify an active reservation changing position, date and time& 2.3.1 Userinterface, Action, Activity, Clientnetworkinterface\newline
											2.3.2 Servernetworkinterface, Controller, Ridesmanager, Rid \newline
											2.1 GPS\\ \hline
      & The system will accept modification only if sent before the taxi allocation&2.3.2 Controller, Ridesmanager, Ride \\ \hline
      & The system will accept date and time modification if it occur at least two hours after the request or/and after the 
	previous reservation& 2.3.2 Controller, Ridesmanager, Ride\\ \hline
      & A taxi driver have the possibility to remove himself from the queue by clicking the:``Disable'' button& 2.3.1 Userinterface, Action, Activity, Clientnetworkinterface \newline
														 2.3.2 Servernetworkinterface, Controller, Ridesmanager\\ \hline
      & The system will remove a taxi from the list if receive the corresponding request by the taxi driver, or if the taxist logged out& 2.3.2 Ridesmanager, Controller\\ \hline
      &  A master terminal interface must be implemented in order to allow, the stakeholder, to configure some parameters 
	 (the number of the taxi zones, the set of positions belonging to each zone, the number of reports (per day, month and year)
	 needed to automatically ban a user and the maximum number of reports ( per hour) a user can insert)& 2.3.3 Master view\\ \hline
      & Every time a report is added to a user, the system will check if the constraints inserted by the master terminal are satified,
	otherwise the system must automatically ban the user&  2.3.2 User\\ \hline
      & The master terminal interface allows to manually ban users, or enable banned users& 2.3.3 Master view\\ \hline
      & The system must refuse reports added by a user if the user has already reached the maximum number of reports ( per hour) decided
	 by the master terminal& 2.3.2 User \\ \hline
      & When the system refuse a report, a notification appear on the user screen, reminding him that he has already exceeded
	the maximum number of reports for that hour& 2.3.1 Userinterface, Activity, Clientnetworkinterface\newline
						     2.3.2 Servernetworkinterface, User, Controller\\ \hline
 \end{longtable}
\end{center}
