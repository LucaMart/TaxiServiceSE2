\documentclass[a4paper,11pt]{report}
\usepackage[utf8]{inputenc}

% Title Page
\title{Requirement Analysis and Specification Document}
\author{}


\begin{document}
\maketitle
\tableofcontents

\chapter*{Introduction}
\addcontentsline{toc}{chapter}{Introduction}
\addtocounter{chapter}{1}

\section{Purpose}
This document aims to describe, specify and analyze the software requirements for \textit{My Taxi Service}. \\
My Taxi Service is needed to provide a passenger-friendly interface to interact with the city's taxi service, and ensure
a fair management of the city-wide taxi deployment.

\section{Scope}
%TODO: (michele) Here I will used some terms that are specific in the context of the project...
% Should I make a reference to the appropriate definition in section "Definitions, Acronyms and Abbreviations"?
% I have added some footnotes though
The passengers should be able to use an application (either mobile or browser based) to request a taxi through the system,
which in turn should answer with the ETA and identification code of the incoming tax. \\
The passengers should be compelled to provide their current location to the system, for their request to be accepted. \\
The taxi drivers should be able to use a mobile application to communicate their availability to the system, and accept
 or refuse incoming calls. \\
The system shall manage a queue of taxis for each taxi zone\footnote{Partition of the city}. \\
The system shall receive GPS location data from each taxi, and use that information to assign each taxi to a 
taxi zone; an available taxi is automatically placed into the taxi queue belonging to the taxi zone it currently occupies. \\
The system shall remove a taxi from the queue upon receiving a confirmation in which the driver accepts an incoming call; a
further confirmation is also sent back to the driver. If a taxi (driver) does, on the other hand, refuse an incoming call, the system
shall move it to the bottom of its taxi queue. \\
The system matches a passenger's position to a taxi zone, and uses that information to forward the call to the first taxi
available in the relative taxi queue. \\
The system shall provide an Application Programming Interface, to make room for future improvements. \\
The system shall also provide the possibility of requesting the reservation of a taxi; said reservation must occur not earlier
than 2 hours before the actual time of the ride; the time of the ride has to be specified by the passenger during the
reservation procedure, as well as the passenger's location and destination. \\
However, the system will actually allocate a taxi (by means of removing it from the queue) only 10 minutes before the
requested time of the ride.
% TODO: (michele) part III
\section{Goals}
The passengers must be able to:
\begin{itemize}
  \item [G1] Transmit its position and the desired destination to the system, thus initiating the Request of a taxi
  \item [G2] Receive the code and the ETA of the incoming taxi
  \item [G3] Reserve a taxi for a certain time  % TODO: this sounds kinda ambiguous, i.e. "this taxi is mine for xx time"
  \item [G4] Request a shared ride
  \item [G5] Receive the fee amount he need to pay in case of a shared ride
\end{itemize}
The taxi driver must be able to:
\begin{itemize}
  \item [G6] Answer a passenger's request
  \item [G7] Render him/herself available to the scheduler
  \item [G8] Render him/herself unavailable to the scheduler
  \item [G9] Receive informations regarding the fee defined for each passenger in case of shared ride
\end{itemize}
The system must be able to:
\begin{itemize}
  \item [G10] Offer a programmatic interface to enable the development of additional services
\end{itemize}
\section{Definitions, Acronyms and Abbreviations}
\begin{description}
  \item[Passenger:] the person who needs to call a taxi.  % TODO: is he considered a passenger even if the request has not yet been initiated?
  \item[System:] the automatic part that manages the service.  % (?) <- add doubts/TODOs as comments pls
  \item[ETA:] estimated time of arrival.
  \item[Taxi zones:] geographical partitions of the city, non overlapping, with an average size of 2Km\textsuperscript{2}.
  \item[Queue:] a list of all available taxis in the corresponding taxi zone. It is managed as a FIFO queue.
    There is exactly one taxi queue associated to each taxi zone.
  \item[GPS:] global position system.
  \item[Shared ride:] a passenger shares the ride with other peoples that origins from the same zone, and go to the same direction
    % TODO: what is "same direction"? Maybe "same destination"?
\end{description}
% WIP
 
\section{Reference documents}

\section{Overview}


\chapter*{Overall description}
\addcontentsline{toc}{chapter}{Overall description}
\addtocounter{chapter}{1}
\section{Product perspective}
The system must interact with a map service, to retrieve information about the route to send to the taxi driver in case of shared ride.
\section{Constraints}
We will develop a unique mobile application that can be used by both passengers and taxi-drivers.\\
The web application will include only passengers functions.\\
The mobile application must be available for Android, Windowsphone and iOS.\\
\section{User characteristics}
The users must be connected to the network to use the application.
Passengers can interact with the service through a web browser or a mobile application.
Taxi driver must access to the application with a device provided of GPS.
\chapter*{Specific requirements}
\addcontentsline{toc}{chapter}{Specific requirements}
\addtocounter{chapter}{1}

\end{document}          
