\section{Product perspective}
The system must interact with a map service, to retrieve information about the route to send to the taxi driver in case of shared ride.


\section{Constraints}
We will develop a unique mobile application that can be used by both passengers and taxi-drivers.\\
The web application will include only passengers functions.\\
The mobile application must be available for Android, Windowsphone and iOS.\\


\section{User characteristics}
The users must be connected to the network to use the application.
Passengers can interact with the service through a web browser or a mobile application; they don't need any particular 
ability or foreknowledge to use it.\\
Taxi driver must access to the application with a device provided of GPS; since they must follow a standard procedure 
they must attend a formation course before starting (2 hours will be enough).\\


\section{Assumptions}
\begin{itemize}
  \item If a request comes from a zone, whose queue is empty, then the system forwards the call to the first taxi 
  in the queue corresponding to an adjacent taxi zone, starting from the northeast.
  \item A passenger is required to subscribe an account to utilize the taxi services (taxi request, taxi booking, taxi sharing)
  \item Taxi drivers can create only one account per vehicle ID
  \item Passengers who reserve a taxi can delete the reservation; if a taxi was allocated for the ride, 
  the system will notify the taxi driver and put him at the top of the queue.
  \item
\end{itemize} 

