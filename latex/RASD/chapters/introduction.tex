\section{Purpose}
This document aims to describe, specify and analyze the software requirements for \textit{My Taxi Service}. \\
My Taxi Service is needed to provide a passenger-friendly interface to interact with the city's taxi service, and ensure
a fair management of the city-wide taxi deployment.


\section{Scope}
%TODO: (michele) Here I will used some terms that are specific in the context of the project...
% Should I make a reference to the appropriate definition in section "Definitions, Acronyms and Abbreviations"?
% I have added some footnotes though
The passengers should be able to use an application (either mobile or browser based) to request a taxi through the system,
which in turn should answer with the ETA and identification code of the incoming tax. \\
The passengers should be compelled to provide their current location to the system, for their request to be accepted. \\
The taxi drivers should be able to use a mobile application to communicate their availability to the system, and accept
 or refuse incoming calls. \\
The system shall manage a queue of taxis for each taxi zone\footnote{Partition of the city}. \\
The system shall receive GPS location data from each taxi, and use that information to assign each taxi to a 
taxi zone; an available taxi is automatically placed into the taxi queue belonging to the taxi zone it currently occupies. \\
The system shall remove a taxi from the queue upon receiving a confirmation in which the driver accepts an incoming call; a
further confirmation is also sent back to the driver.
 If a taxi (driver) does, on the other hand, refuse an incoming call, the system
shall move it to the bottom of its taxi queue. \\
The system matches a passenger's position to a taxi zone, and uses that information to forward the call to the first taxi
available in the relative taxi queue. \\
The system shall provide an Application Programming Interface, to make room for future improvements. \\
The system shall also provide the possibility of requesting the reservation of a taxi; said reservation must occur at least
2 hours before the actual time of the ride; the time of the ride has to be specified by the passenger during the
reservation procedure, as well as the passenger's location and destination. \\
However, the system will actually allocate a taxi (by means of removing it from the queue) only 10 minutes before the
requested time of the ride.
% TODO: (michele) part III


\section{Domain properties}
In this section we will analyze the background laying behind My Taxi Service:
\begin{itemize}
  \item Passengers are assumed to be reliable and trustworthy, meaning that if they request a taxi or make a reservation
  they will declare their real position and actually use the taxi
  \item Passengers will pay at the end of the ride the amount of money demanded by the taxi driver
  \item Taxi drivers must own a valid taxi driving license
  \item ETA is estimated with a maximum error margin of 5 minutes
  \item We assume that, for each zone, if its queue is empty, then at least one queue in the eight adjacent zones has available taxis
  \item We assume GPS coordinates reliable
\end{itemize}
\section{Goals}
The passengers must be able to:
\begin{itemize}
  \item [G1] Transmit its position and the desired destination to the system, thus initiating the Request of a taxi
  \item [G2] Receive the code and the ETA of the incoming taxi
  \item [G3] Reserve a taxi for a time period, starting at the time specified during the reservation\footnote{the starting of the
  reserved ride must occur at least 2 hours after the time of the reservation}, and ending after the ride is complete.
  \item [G4] Request a shared ride
\end{itemize}
The taxi driver must be able to:
\begin{itemize}
  \item [G5] Answer a passenger's request
  \item [G6] Render him/herself available to the scheduler
  \item [G7] Receive informations regarding the fee defined for each passenger in case of shared ride
\end{itemize}
The system must be able to:
\begin{itemize}
  \item [G8] Offer a programmatic interface to enable the development of additional services
\end{itemize}


\section{Definitions, Acronyms and Abbreviations}
\begin{description}
  \item[Passenger:] the user who sends a taxi request.
  \item[User:] an human interacting with the system. Users are split in 2 classes: `passengers' and `taxi drivers'.
  \item[System:] the automatic part that manages the service.
  \item[ETA:] estimated time of arrival.
  \item[Taxi zones:] geographical partitions of the city, non overlapping, with an average size of 2Km\textsuperscript{2}.
  \item[Queue:] a list of all available taxis in the corresponding taxi zone. It is managed as a FIFO queue.
    There is exactly one taxi queue associated to each taxi zone.
  \item[GPS:] global position system.
  \item[Shared ride:] a passenger shares the ride with other people that origins from the same zone, and go to the same direction
    % TODO: what is "same direction"? Maybe "same destination"? <--- direction was mentioned in the Assignments, we have to contact the customer
  \item[Active reservation]: a reservation is considered active if it is in the reservation scheduler and isn't occured yet
    \begin{center}
    •
    \end{center}
\end{description}
% WIP


\section{Reference documents}


\section{Overview}

