
\section { Functional Requirements}
 
For each goal, we define the specific function that we will have to implement;\\
\begin {itemize}
\item [G1]:
	\begin{itemize}
	\item Users can create an account;
	\item Users can log into their account;
	\item Passengers can select from a menu the option of requesting a taxi as soon as possible; 
	\item Passengers can insert their position filling an input form and confirm it;
	\item The system will receive the request and identify the zone in which the passenger is in;
	\item The system will forward the request to the first taxi in the selected zone queue and wait for an answer;
	\item If the taxist accepts, the system will remove him from the queue; otherwise it will append the taxist to the last position and
scan the list for a taxist to accept;
	\end{itemize}
\item [G2]:
	\begin{itemize}
	\item As soon as a taxist accepts a request, the system invokes the support system to calculate the ETA giving the position of the taxi and the position of the passenger;
	\item The system will communicate the taxi code and the ETA;
	\end{itemize}
\item [G3]:
	\begin{itemize}
	\item Passengers can select from a menu the option of reserving a taxi for a chosen ride and date; 
	\item Passengers can insert the initial and final position, time and date, their email and confirm it;%account or not?<---we need to talk about this, cause withouth an account, some villain could request  taxis withouth taking it, taxi-drivers may react like --> >.< fuck this technology! #?!*§!
	\item The system will receive the reservation and if it respects the 2 hour constraint it will send a confirmation;
	\item Ten minutes before the ride starts, the system allocates a taxi for it.
	\end{itemize}
\item [G4]:
	\begin{itemize}
	\item The application must have a selectable option labled:"share your ride", that allows passengers to enable the shared ride service. In case of non reserved ride, the application will ask passengers the amount of time they can wait for others people.
	\item When the system receive a request of a shared ride, it will search for others shared ride requests starting from the same taxi zone, and going in the same direction. %TODO decide by ourselves or ask  the "customer" what does "same direction" mean
	\item When a new passenger is added to a shared ride, the system will interact with the map service, in order to retrieve a new route for the taxi driver, and to calculate new fees
	\item When the timeout of one passengers ,added to the current ride, occur, the system will procede with the allocation of the taxi .
	\item After the taxi allocation, the passengers who requested the shared ride will receive, not only the taxi ID, but also the fee they have to pay.
	\end{itemize}
\item[G5]:
	\begin{itemize}
	\item When a request comes or there is a reservation for the next 10 minutes without an assigned taxi driver, the system must search for the first queued taxi and forward him the request.
	\item If a taxi driver refuse to take care about a call, the system will move him at the end of the queue, and forward the request to the next taxi driver in the queue. If a queue is empty, the system will search a taxi driver in a queue belonging to an adiacent taxi zone % What all taxi driver refuse to take care of a certain call, infinite loop?.....and also: If there are more than one adiacent zones, how to decide which to choose?--> in the domain properties we assume that there is at least one.. 
	\end{itemize}
\item [G6]:
	\begin{itemize}
	\item A taxi driver logged in into the system can select the button " Ready ", then the system will notify the system that the logged user is ready to accept some passengr's call. The application also send the taxi driver's postion detected with a GPS
	\item If the application needs to retrieve data from a GPS and this isn't available, it will remind  the user to turn it on.
	\item When the system receive a notification , by a taxi driver, informing that he is ready to take care of some passengers, it will append the user in the queue corresponding to the taxi zone that include the position retrieved by the application.
	\end{itemize}
\item[G7]
	\begin{itemize}
	\item When a taxi driver is assigned to a shared ride, the system will send him the route he need to follow, and the fee amount every passenger have to pay
	\end{itemize}
\item[G8]:
	\begin{itemize}
	\item
	\item
	\end{itemize}
\end {itemize}
\begin{itemize}
 \item Passengers can access a section, in which they be able to check the ID of the taxi assigned to their ride and manage ( delete or modify) an active reservation. %TODO specify the meaning of ACTIVE RESERVATION
 \item When a passenger delete a reservation, the system will remove it from the reservation scheduler and, if a taxi driver is already assigned, notify the taxist.
 \item A passenger can modify an active reservation changing position, date and time.
 \item The system will accept modification only if sended before the taxi allocation.
 \item The system will accept date and time modification if it occur at least two hours after the request or/and after the previous reservation.
 \item A taxi driver have the possibility to remove himself from the queue by clicking the:``Disable'' button.
 \item The system will remove a taxi from the list if receive the corresponding request by the taxi driver, or if the taxist logged out for XXXXX minutes. %TODO decide how much or decide to remove him when we need to send him a request
\end{itemize}

 • • •
                                                                                                                                                                                                                                                                                                                                                                                                                                                                                                                                                                                                                                                                                                                                                                                                                                                                                                                                                                                                                                                                                                                  
