\section{Entry criteria}
Due to start an integration test two constraints must be satisfied:
the major classes must be covered by ,at least , a 60 percent of unit test, 
while for the others a 30 percent is sufficient.
%TODO spiegare meglio quali siano le classi 'major'
%Ogni classe deve avere una percentuale di unit test del 60% per le classi principali,
%e il 30% per quelle derivate.

\section{Elements to be integrated}
In our case element is synonym of class; now we're going to show the classes that
need integration test in order to be sure that our application will work correctly.
\begin{itemize}
  \item[Ridesmanager]: it needs to be integrated with:
    \begin{itemize}
      \item[Ride, Sharedride]: in order to store information about the actived rides
      \item[Taxiqueue]: in order to take information of available taxis
        in case of taxi request.
      \item[Controller]: in order to exchange information about user's( and also guest's) requests
    \end{itemize}
  \item[Controller]: it needs to be integrated with:
    \begin{itemize}
      \item[User]: in order to create an ad-hoc Controller and to retrieve information about users
      \item[Servernetworkinterface]: in order to communicate with the corresponding client side
    \end{itemize}
  \item[Servernetworkinterface]: it needs to be integrated with:
    \begin{itemize}
      \item[Clientmessage]: in order to read client's messages
      \item[Servermessage]: in order to send messages to the client
    \end{itemize}
   \item[Activity]: it needs to be integrated with
     \begin{itemize}
       \item[Action]: in order to provides the allowed actions
       \item[Userinterface]: in order to provide the set of items this class needs to show
     \end{itemize}
   \item[Action]: it needs to be integrated with the Clientnetworkinterface in order to send
     requests to the server
   \item[Userinterface]: it needs to be integrated with the Clientnetworkinterface in order to
     show the right Activity according to the server message
   \item[Clientnetworkinterface]: it needs to be integrated with:
     \begin{itemize}
       \item[Clientmessage]: in order to send messages to the server 
       \item[Servermessage]: in order to read server's messages
     \end{itemize}
\end{itemize}
%Ridesmanager: Sharedride, Ride, Controller:{ User, Servornetwrokinterface },
%              Taxiqueue
%Servernetworkinterface: Clientmessage, Servermessage
%Activity: Action:{Clientnetworkinterface},Userinterface 
%Clientnetworkinterface: Clientmessage, Servermessage

\section{Integration testing strategy}
In this section we will explain how we plan the integration test
in order to build, as soon as possible, a running application 
with few working features; this will allow us to easly show
our progress to the customer, and also , in case of delay, 
to launch a working application, also with missing requirements.
In order to reach our goal we decide to apply a bottom-up method
for integration test and top down method for unit test.
%Unit testing->Top-down
%Integration testing->bottom-up

\section{Sequence of component/Function integration}
%1- Componenti base del server: Ridesmanager, Controller, Guestcontroller
%2- Componenti base del client: Action, LoginAction, Activity, Userinterface
%3- Componenti base networking: Clientnetworkinterface Servernetworkinterface,
%                               Clientmessage, Servermessage
%4- Integrazione server-client
%5- Componenti utente: User, Taxidriver, Passenger, Passengerscontroller
%                      Taxidriverscontroller, Ride
%6- Integrazione utente
%7- Fuznioni avanzate: Sharedride, Reservation
%8- Integrazione funzioni avanzate

\subsection{Software integration sequence}
%punti: 4, 6, 8

\subsection{Subsystem integration sequence}
%punti: 1,2,3,5,7

